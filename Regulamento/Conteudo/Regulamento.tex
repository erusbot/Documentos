%************************************************
\chapter{Da Natureza das Finalidades}
\label{chp:natureza}
%************************************************

A Equipe de Robótica da Universidade Federal do Espírito Santo (ERUS) é composta por alunos da própria universidade, com intuito de promover a robótica na instituição, bem como melhorar a qualidade de seus respectivos cursos. Para isso a equipe objetiva participar de competições de robótica nacionais e internacionais, bem como adquirir conhecimento através de pesquisas para o desenvolvimento de artigos, minicursos, palestras e outras atividades para o meio acadêmico. Por fim, a equipe também deve aproximar a sociedade com a robótica, promovendo palestras, apresentações, competições e outros eventos que contribuam para esse fim. 


%************************************************
\chapter{Hierarquia da Equipe}
\label{chp:hierarquia}
%************************************************

A equipe se organizará de forma horizontal. Para a execução de projetos, comissões serão formadas e se organizarão de forma orgânica. Entretanto, para garantir a coesão da equipe as seguintes atribuições serão feitas:

\begin{itemize}
  \item[] \textbf{Líder} - Responsável pelo gerencimanto da equipe. Cuidará das questões burocráticas, coordenação das reuniões internas e participação nas externas. Ficará a cargo da aplicação de penalidades e demais funções que tange a um líder. 
  \item[] \textbf{Vice-Líder} - Deverá assumir o posto de líder caso esse esteja impossibilitado de fazê-lo.
  \item[] \textbf{Gestor de Finanças} - Administrará todo o dinheiro arrecadado ou gasto pela equipe. Para isso deverá apresentar um balanço à equipe sempre que houver movimentações.
  \item[] \textbf{Gestor de Comunicação} - Coordenará as atividades no site e redes sociais da equipe, bem como a divulgação dos eventos promovidos pela mesma.
\end{itemize}


%************************************************
\chapter{Atividades da Equipe}
\label{chp:atividades}
%************************************************

\begin{itemize}
\item Comissões serão estruturadas para atender a demanda de cada atividade. A quantidade de membros envolvidos será dado de acordo com as necessidades das tarefas.

\item Cada comissão possuirá um integrante encarregado de gerenciá-la e reportar o progresso das atividades à equipe. Este poderá criar sub-comissões para subdividir as tarefas. O encarregado será responsável pelo bom funcionamento da comissão, cabendo a ele a delegação de atividades e cobrança de cumprimento.

\item Cada comissão poderá ter seu próprio calendário de tarefas e reuniões, desde que esteja em harmonia com as demais tarefas da equipe.

\item O não cumprimento das tarefas estabelecidas na comissão estará sujeito às penalidades previstas no \autoref{chp:penalidades}
\end{itemize}


%************************************************
\chapter{Categorias de Integrantes}
\label{chp:categorias}
%************************************************

Haverá três modalidades de participação na equipe, sendo elas:

\begin{itemize}
\item[] \textbf{Bolsista} com carga horária semanal de \textbf{1200 minutos} (20 horas).
\item[] \textbf{Voluntário} com carga horária semanal de \textbf{600 minutos} (10 horas).
\item[] \textbf{Petiano} com carga horária semanal estipulado de acordo com o plano de carreira do PET, com um mínimo de \textbf{300 minutos} (5 horas) semanais.
\end{itemize}

Membros com suficiente tempo de participação na equipe receberão certificado com carga horária condizente com a sua categoria. A distribuição de carga horária deverá ser feita de forma a maximizar a sinergia dos membros, escalações de horário com menos de 3 integrantes presentes poderão ser vetadas.


%************************************************
\chapter{Encargos dos Integrantes}
\label{chp:encargos}
%************************************************

Independente da categoria do integrante, esse deverá cumprir os seguintes encargos:

\begin{itemize}
\item Cumprimento da carga horária estabelecida no \autoref{chp:categorias}
\item Realização das atividades atribuídas
\item Confeção de um artigo (tutorial) por mês
\item Participação em ao menos uma das comissões de competições
\end{itemize}

A contabilização de carga horária cumprida será feita de forma automática, portanto estará sujeita a desvios. Uma margem de erro de 10\% para mais será aplicada por padrão, caso ainda assim a meta não seja cumprida e o integrate venha a recorrer, será feita uma análise dos dados (tarefas entregues, \textit{commits} nos repositórios, etc).

O não cumprimento das atividades estará sujeito às penalidades dispostas no \autoref{chp:penalidades}. 

%************************************************
\chapter{Reuniões da Equipe}
\label{chp:reunioes}
%************************************************

A equipe realizará reuniões para deliberações formais acerca de questões de organização, estratégias e planejamento da equipe. Essas ocorrerão de acordo com a necessidade e em horário de comum acordo. Reuniões internas de comissões não deverão conflitar com reuniões gerais da equipe.

Durante a reunião haverá um integrante encarregado de confeccionar a ata de reunião e outro para o direcionamento dos pontos. Integrantes com conflito de horário deverão obrigatoriamente ler a ata posteriormente e se dirigir ao líder em caso de dúvida em relação a algum ponto.

A tolerância de atraso para a reunião será de 15 minutos. Após isso, integrantes que chegarem até 30 minutos após o início da reunião serão considerados atrasados, passados 30 minutos são considerados ausentes. Dois atrasos constituem uma ausência e as penalidades associadas estão dispostas no \autoref{chp:penalidades}.


%************************************************
\chapter{Utilização de Recursos da Equipe}
\label{chp:recursos}
%************************************************

A sala da equipe estará à disposição dos integrantes a qualquer momento. A utilização dela e dos recursos nela presentes por terceiros é autorizada desde que não interfira nas atividades de algum integrante da equipe.

O empréstimo de materiais para integrantes é permitido desde que autorizado pelo líder e informado ao e-mail do grupo. O integrante fica responsável pelo material emprestado, tendo que repor o mesmo em caso de danos.

A organização da sala é responsabilidade de todos os integrantes. Dada a necessidade, mutirões de limpeza poderão ser realizados e haverá participação compulsória de todos os integrantes.

%************************************************
\chapter{Penalidades}
\label{chp:penalidades}
%************************************************
As penalidades têm por finalidade o incentivo ao cumprimento das atividades e regras previstas neste documento e deverão ser aplicadas de acordo com a necessidade. Penalidades serão aplicadas na forma de advertências, o ocorrimento de duas advertências  no período da atividade equivale a uma multa no valor de R\$ 10,00 ou 10\% do valor da bolsa (para integrantes bolsistas).

As penalidades deverão ser aplicadas seguindo esta orientação:
\begin{itemize}
\item[] Ausência na reunião: 1 advertência.
\item[] Atraso no cumprimento de atividade: 1 advertência.
\item[] Não cumprimento de carga horária: 1 advertência.
\item[] Não cumprimento de atividade: 1 multa.
\end{itemize}
Na presença de justificativa plausível as penalidades poderão ser abolidas. Em casos extraordinários o líder poderá determinar a penalidade.

Caso um membro venha a receber penalidades recorrentemente, esse estará sujeito ao desligamento da equipe e perda da bolsa (em caso de bolsista).


%************************************************
\chapter{Regimento}
\label{chp:regimento}
%************************************************

Este documento está sujeito a alterações que deverão ser aprovadas em reunião semanal.